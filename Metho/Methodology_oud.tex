\chapter{Methodology}

This research study is qualitative in nature and this chapter focuses on the research methodology and design.
 
Amongst all philosophical assumptions reviewed, the interpretive paradigm has been identified as the most appropriate framework to use for this study. 

Secondly, the research design will be discussed of which qualitative research method is used in the form of multiple case study which will be explored to get the required information for this study. Lastly the research will discuss the case criteria selection and the data collection methods.

\section{Multiple Case Study}

To collect the data for this study the method of the multiple case study has been selected. 
The multiple case study has been seen as the ideal method of collecting data and or information for this research question since we are looking at two different regions of the world based on the economic development of that area (\gls{EE} vs. \gls{DE}) and also researching two different industries in these economic areas, Pharma and ICT services.\\

Qualitative case study methodology provides tools for researchers to study complex phenomena within their contexts.
It is an approach to research that facilitates exploration of a phenomenon within its context using a variety of data sources.
This ensures that the issue is not explored through one lens, but rather a variety of lenses, which allows for multiple facets of the phenomenon to be revealed and understood (Baxter and Jack, 2008)

\section{Case criteria and selection}

\begin{figure}
\centering
\begin{tikzpicture}[scale=0.85, transform shape]
% STYLES
\tikzset{
    ellipsXL/.style={rectangle, draw, 
    align=center,minimum width=12.5cm,minimum    
    height=8.0cm,>=stealth}%fill=black!10
}

\tikzset{
    ellipsM/.style={ellipse, draw, minimum width=100pt,
    align=center,node distance=3cm,inner sep=5pt,text width=4.0cm,minimum    
    height=3.5cm,>=stealth}%fill=black!10
}

\tikzset{
    ellipsS/.style={ellipse, draw, minimum width=100pt,
    align=center,node distance=3cm,inner sep=5pt,text width=1.5cm,minimum    
    height=1.5cm,>=stealth}%fill=black!10
}

\tikzset{
    ellipsL/.style={ellipse, draw, minimum width=100pt,
    align=center,node distance=3cm,inner sep=5pt,text width=6.5cm,minimum    
    height=6cm,>=stealth}%fill=black!10
}

\node [ellipsXL](World)[label=World] {};
\node [ellipsL](Econ) [label=Economies]{};
\node [ellipsM] (Ind)[label=Industries] {};
\node [ellipsS] (Firms)[label=Firms] {};

% Draw the links between 
%\path[<->,thick] 
 %  (Choices) edge node[anchor=center, text width=3.5cm, below left, midway] {Industry conditions and 
%    firm-specific resources}  (Institutions) 
%   (Choices) edge node[anchor=center, text width=3.5cm, below right, midway] {Formal and informal 
%   constraints} (Organisations)
% (Institutions) edge  node [midway, below] {interaction} node[midway, above] {Dynamic} (Organisations);
\end{tikzpicture} 
\caption[Levels of differentiation]{Levels of differentiation. Source: Author}%
\label{fig:Levels} 
\end{figure}


The next step is to identify and select the cases that will be used. 
As mentioned by \cite{Pettigrew:1990}, the number of cases that are studied is usually limited, therefore it is a good approach to select cases that signify correctly the differences at hand.\\
Viewed from a global point of view, the influence of the  \wto~ spans the entire globe. This playing field is than decided into two economic blocks: \gls{EE} and \gls{DE}. As these two blocks span a great number of countries and continents, India (a \gls{EE}) and the \gls{EU} have been chosen to represent the two economic blocks. Both India and the EU have an equally long standing relation with the \wto~and have been active in the various life cycles of the \wto~ for a great number of years.\\
The next level of distinction that can be determined is the industry level. Within both the EE and DE one can find different industry types. \cite{Porter:1980} already identified the importance of industries to the search for \gls{CA}. 
The industry types have to be existent in sufficient quantaties in both economic blocks.  


