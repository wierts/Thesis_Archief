\chapter{Discussion}
This chapter discusses to what extend the working propositions derived in~\ref{ch:LitReview} are supported by the outcome of the analysis, that was done in chapter~\ref{ch:Result}.
The outcome gives an indication to if and to what extend the theoretical findings can be applied to real-live situations.
This can give an insight as to the influence of institutions such as the WTO on \glsfull{AE} respectively \glsfull{EE}.

The within-case analysis provided insight in the responses of similar firms in the two industries across both economic regions.
The firms in the European economic area have shown similar or almost similar responses to the \cc that came from the \rr that are imposed by the WTO\@.
The only dissonant behaviour has been observed by Cipla.
They are actively resisting the new \rr (Patent Laws) and engaged in multiple lawsuits because of this stance.
As a sample for the entire Indian \pharma industry, the stance of Cipla could be atypical.
The claim that \textit{Firms in the same economic region and in similar industries (Services or Manufacturing) are likely to have a homogenise response to changes in the institutional environment} (WP1a) is partially supported.

Different Industries face different issues at any given time.
This is reflected in the different responses by firms at that given time.
Indian \pharma firms are copying the strategy it the \its firms to some extend. 
Only this strategy was employed by the \its firms some 10 years ago.
At present they have very different problems and thus responses.
The element of time is very important when judging the responses.\\
\its firms from both economic regions faces similar problems and choose somewhat identical responses.
All four companies have gone through leadership changes in the last 3 years. 
All are or have been reorganising in more or less similar degree.
Here the maturity of the companies (and industry) might also be a factor.
An argument could be made that the \pharma industry in Europe is more mature\footnote{Company maturity has not been a factor in this research and no literature has been consulted on this topic, when making any statements on this part}.
Maturity here, is based on the number of years the companies have existed.
Significant differences were found between the responses of European \pharmas and Indian \pharmas.
As fas as such a claim can be made,  maturity of the firms, relative to the industry as a whole, is suspected to have an influence.

Especially the \its companies gave support for the claim that multiple embeddedness can have a homogeneous effect on firm responses.
The fact that all companies are truly global enterprises, all four are on the same playing field when it comes to the services (or IT products) they provide. 
This is not the case in the \pharma industry.
The \glsfull{AE} firms are far bigger than the \glsfull{EE} firms in the \pharma industry.
This size difference creates a very different playing field.
The analysis of the four \pharma companies does not support this claim that multiple embeddedness can have a homogeneous effect on firm responses.

The changes in \rr that occurred during the period under investigation, all had a radical and fundamental impact on the industries under investigation.
The WTO \rr can be considered second order changes as defined by~\cite{Meyer:1995td}.

That firm history is in important influence on the type of response a company chooses has been observed in the European \pharma industry.
As a matter of automatic response the \pharma start to \acq other companies when revenues tend to go down.
The majority of the companies that are considered `Big Pharma' have been the results of mergers (GSK, Pfizer, Sanofi, and AstraZeneca).
The press has linked Novartis and Roche (both Swiss) more than once in merger rumours~\citep{The-New-York-Times:2011}.
The companies in the other industries have not disproven this claim, however no indication has been found to contradict this claim either. 
This working proposition is therefor supported.

\Glsdesc{AE} firms, especially in the \pharma industry have faced almost no institutional change.
The effect of the change they have seen, only provided them with opportunities in outsource and offshore certain types of works (o.a. \gls{crams} and clinical trails).
The \glsfull{EE} have seen a lot of change in the institutional environment.
The adoption of \gls{trips} by India is the foremost example.
This effectuated a change in the mannor they had to do business.
The \gls{EE} \its firms did see some institutional change.
This was not initiated by the WTO, it was the public opinion that, especially in the US, turned negative towards offshoring work and thus shipping jobs to other countries.
The European \its companies did not face this kind of critique.
They also had work done by people in low wage economies.
Working Proposition~\ref{WP:AE_lower_inst_cha} is thus supported.

Working Proposition~\ref{WP:greater_diff_heterogeneous_responses} is also supported by using the same reasoning as above and considering the following.
The institutional difference in the \rr between the \glsfull{AE} \pharma firms and the \glsfull{EE} \pharma firms is far greater than of the \its firms.
This resulted in very different response to these institutional changes.
While the Indian \its firms merely started a PR campaign, the Indian \pharma firms could no longer revert to reverse engineering drugs.

The modes by which firms choose to adjust is not necessarily heterogeneous across industries.
All firms have acquired other firms, either for growth or diversification purposes. 
No clear division has been observed between the \pharma and \its industries in terms of response choices.
Therefor \wpro has been rejected.

Though it is difficult to determine the effects of internal resources through the analysis of newspapers articles, the effects of chairman and CEO's has been found in these sources.
The CEO's of Ranbaxy and Cipla both have been quote in Indian newspapers.
Their influence can be said to be quite high.
The \rbt by~\citep{Barney:2011jp,Barney:1991ur} considers all  resources in the company.
Here the only resources that have been considered are human beings.
The analysis of the other firms did not result in such explicit differences in firm behaviour to similar changes in (institutional) \rr.
Considering the difference between Cipla and Ranbaxy is the only well founded example, \wpro~\ref{wp:rbt}is considered at least partially supported.
It is not fully supported due to the lack of additional evidence.  


\begin{table}
\centering
\caption{Working Propositions and level of support. Source: Author}\label{tab:support4WP}\renewcommand{\arraystretch}{1.333}
    \begin{tabular}{p{2.06cm}p{9.5cm}p{1.8cm}}
%& \multicolumn{2}{c}{European}& \multicolumn{2}{c}{Indian}\\
    \textbf{Working Proposition}  &\centering \textbf{Description}  &  \textbf{Findings}\\
     
\toprule
\textbf{WP 1.A} &%
     Firms in the same economic region and in similar industries (Services or Manufacturing) are likely to have a homogenise response to changes in the institutional environment  %
                                                     &  Partially Supported         \\
\textbf{WP 1.B} &%
    Firms in the same economic region, but in different industries (Services or Manufacturing) are likely to have a heterogenous response to changes in the institutional environment%                                   
                                                    & Supported                       \\
\textbf{WP 1.C} &
    Firms in the same industry (Services or Manufacturing) but in different economic regions are expected to have a heterogenous response to changes in the institutional environment  
                                                    & Supported                      \\
\textbf{WP 2} &
    Multiple embeddedness is expected to have a homogenising effect on the responses of firms in similar industries across different economic regions.                          
                                                    & Supported       \\
\textbf{WP 3}&
    The WTO rules and regulations are expected to incur only second order changes        
                                                    &Rejected       \\
\textbf{WP 4}  &%
    Firms are more likely to adopt to change in a manor they have grown accustomed to over time (they use strategies they have used in the past)
                                                    & (Partially) Supported              \\
\textbf{WP 5.A} &%
    Advanced economy firms are expected to face a lower degree of institutional change compared to emerging economy firms with respect to (changes in) WTO rules and regulations                                         
                                                     & Supported                \\
\textbf{WP 5.B}  &%
    The greater the difference in institutional change between regions in response to WTO rules and regulations the more likely heterogeneous responses are from firms in the same industry in their respective regions  
                                                    & Supported             \\
\textbf{WP 6}  &%
    The preferred mode of adjustment to WTO rules and regulations for MNE in different industries is likely to be heterogeneous     
                                                    & Rejected                      \\
\textbf{WP 7}  &%
    The diversity in internal resources (humans and knowledge) within firms could be responsible for the heterogeneity of firms responses to changes institutional environment                                    
                                                     & Supported              \\
 \bottomrule                                                                                                                                                                                                                                                   
    \end{tabular}
\end{table}

%\input{Discussion/Table_Cross}