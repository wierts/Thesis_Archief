\section{Firm responses}\label{sec:firmResponses}

\subsection{Responses to institutions}
As seen \inth in Section~\ref{sec:InstTheory} is a powerful tool.
In order to understand the influence of the \wto~on IB in the region of \gls{AE} and \gls{EE} in particular it is of importance to understand how firm react to `institutional change'.
According to~\cite{North:1990vl} institutional change may result from changes in the character and content of either or both of these, or their relevant enforcement mechanisms.

\subsubsection{First and second order changes}
Literature makes a distinction in first and second order changes when it comes to firm responses~\citep{Meyer:1995td}.
First order changes involve changes in processes, systems and or structures. These changes happen in periods of relative calm and tend to span extended periods of time~\citep{Dutton:1991gk,FoxWolfgramm:1998vu,Tushman:1985}.\\
Second order change however is associated with more radical, transformational and fundamental change.
It alters the business organisation at it's core~\citep{Meyer:1982ug,Meyer:1995td,Tushman:1985,Newman:2000fc}.
This leads to the next working proposition.

\begin{WP}\label{WP:WTO_rr_2nd-change}
The WTO~\rr are expected to incur only second order changes 
\end{WP}
Traditionally firms have to respond to changes in the business landscape due to market driven changes~\citep{Chittoor:2008cj}.\\
Now the change is initiated by institutions and this can lead to these aforementioned second order changes~\citep{Chittoor:2009jh}.
So in this case, it is not the market that forced the change, however is the market that, as a whole, has to adapt to the (same) change.
There is a relative paucity of research on how organisations transform themselves in the face of institutional transitions~\citep{Chittoor:2008cj}.\\

\subsubsection{Strategic Responses}
\cite{Cantwell:2009hg} provides a very intriguing insight in the different coping mechanism that firms employ in the face of institutional change.
According to~\cite{North:1990vl} institutional change may result from changes in the character and content of either or both of these, or their relevant enforcement mechanisms.
Among scholars notable contributions on the conception of how firm react to changes have been make by~\citep{Oliver:1991tm,Cantwell:2009hg}.
The broadest being~\cite{Oliver:1991tm} as seen in table~\ref{tab:Oliver:1991}, gives five basic reaction types.
The reactions are categorised from from passivity to increasing active resistance: acquiescence, compromise, avoidance, defiance, and manipulation~\citep{Oliver:1991tm}.

%%%%%%%%%%%%%%%%%%%%%%%%%%%%%%%%%%%
\begin{table}[htdp!]
  \caption[Strategic responses to institutional processes]{Strategic responses to institutional processes (source~\cite{Oliver:1991tm})}\label{tab:Oliver:1991}
\centering
\begin{tabular}{lll} 
  \toprule
	Strategies & Tactics & Examples \\ 
	\midrule
	          &Habit        &Following invisible, taken-for-granted norms\\
Acquiesce     &Imitate      &Mimicking institutional models\\
	          &Comply       &Obeying rules and accepting norms\\
	\\
	          &Balance       &Balancing the expectations of multiple constituents\\
Compromise    &Pacify        &Placating and accommodating institutional elements\\
	          &Bargain       &Negotiating with institutional stakeholders\\
	\\
	          &Conceal       &Disguising nonconformity\\
Avoid         &Buffer        &Loosening institutional attachments\\
              &Dismiss       &Ignoring explicit norms and values\\
    \\
              &Escape        &Changing goals, activities, or domains\\
Defy          &Challenge     &Contesting rules and requirements\\
              &Attack        &Assaulting the sources of institutional pressure\\
   \\  
              &Co-opt        &Importing influential constituents\\
Manipulate    & Influence    &Shaping values and criteria\\
              &Control       &Dominating institutional constituents and processes\\
	\bottomrule
\end{tabular}
\end{table}
%%%%%%%%%%%%%%%%%%%%%%%%%%%%%%%%%%%

Organisations commonly accede to institutional pressures~\citep{Oliver:1991tm}, however there are some instances where an alternative is followed. 
Some forms relevant to the thesis at hand will be discussed here.\\
Imitate (as a form of acquiescence) can be seen as consistent with the concept of mimetic isomorphism, refers to either conscious or unconscious mimicry of institutional models, including, for example, the imitation of successful organisations and the acceptance of advice from consulting firms or professional associations~\citep{DiMaggio:1983wt}.
Compliance is considered more active than habit or imitation, to the extent that an organisation consciously and strategically chooses to comply with institutional pressures in anticipation of specific self-serving benefits that may range from social support to resources or predictability~\citep{DiMaggio:1983wt,Meyer:1978if,Pfeffer:2003wp}.

%%importeren van de referenties en dit koppelen aan isomorphisms en artikel van lawton over firm responses naar de WTO
%Eerst breed dan smaller
Firms hardly follow the changes willingly.  
Under certain circumstances, organisations may attempt to balance, pacify, or bargain with external constituents,~\citep{Oliver:1991tm} they might seek a compromise in the (enforced) changes. 
This might be expressed by behaviours such as balancing. 
Here balancing is meant as the organisational attempt to achieve parity among or between multiple stakeholders and internal interests~\citep{Oliver:1991tm}.
An even stronger negative response can be found in the form of avoidance. \\
Avoidance is defined here as the organisational attempt to preclude the necessity of conformity; organisations achieve this by concealing their nonconformity, buffering themselves from institutional pressures, or escaping from institutional rules or expectations~\citep{Oliver:1991tm}.
An even more active form of resistance is defiance.
The outright manipulation, or exerting power, actively to change or exert power over the content of the expectations themselves or the sources that seek to express or enforce them are two forms of resistance~\citep{Oliver:1991tm}.
These are considered to be not in scope for the purpose of this thesis.
\cite{Cantwell:2009hg} provides a very intriguing insight in the different coping mechanisms that firms employ in the face of institutional change.
He formulates the distinction that firms can opt for three basic responses to changes in the institutional environment.

\begin{itemize}
  \setlength{\itemsep}{5pt}
\item Avoidance
\item Adaption
\item Coevolution
\end{itemize}

Typically avoidance of the institutional rules takes place in weak institutional environment, characterised by a lack of accountability and political instability, poor regulation and deficient enforcement of the rule of law~\citep{Cantwell:2009hg}.
Firms tend to take the (external) institutional environment as a given~\citep{Cantwell:2009hg}. 
The attitude of some Indian \pharma companies and the Indian government in the 1990's towards \gls{IP}can be seen as modern avoidance techniques~\citep{Chittoor:2009jh,Times:2013}.\\
The second form of witch~\cite{Cantwell:2009hg} speaks is institutional adaptation. 
With institutional adaptation the \mne seeks to adjust its own structure and policies to better fit the environment. 
This can be done using the techniques of political influence, bribery or emulate the behaviour, commercial culture and institutional artefacts that are most desirable in that country. 
This line of thought is related to what~\cite{Oliver:1991tm} refers to as imitate (as a form of acquiescence) and later is also referred to as~\iso.
These two forms of firm responses can be considered exogenous. \\
The final form of adaption is partly endogenous in that the~\mne is engaged in a process of coevolution.
Here the objective of the firm is no longer simply to adjust, but to affect change in the local institutions, be they formal or informal~\citep{Cantwell:2009hg}.
The process of coevolution can only take place in the ``Enforcement of Rules Phase'' (see table~\ref{tab:lawton:2009} on the different life cycle phases of the \wto). 
In the implementation phase the \mne (through their home country) cannot influence the rules setting process.
Off course the~\mne can try to influence the proceedings in the first phase (see section~\ref{sec:WTO:formulation_phase}). 


\subsubsection{Different responses in economic regions}
Once a set of rules and regulations has been agreed upon and phased-in in the different (member) countries, the effect that these rules have, varies significantly between~\gls{AE} and~\gls{EE} countries~\citep{Seligman:2008,Shenkar:2006}. 
It is also the heritage of the firm that will contribute to the kind of response that the company will choose~\citep{Carney:2003un}.\\
It are especially the \gls{EE} firms that choose a defensive strategy or exit (fail) in their response to institutional changes~\citep{Chittoor:2008cj}.
This response of fighting the changing environment can be attributed to the administrative heritage of firms in \gls{EE}~\citep{Bartlett:1989vl,Carney:2003un}.\\
The observations by~\citep{Bartlett:1989vl,Carney:2003un,Chittoor:2008cj} lead to the following working 
proposition.

 \begin{WP}\label{WP:history}
 Firms are more likely to adopt to change in a manor they have grown accustomed to over time (they use strategies they have used in the past).
\end{WP}
The role of institutions is more salient in emerging economies because the rules are being fundamentally and comprehensively changed, and the scope and pace of institutional transitions are unprecedented~\citep{Peng:2003uh}.
This mainly has to do with the fact that in~\gls{EE} the domestic rules and regulations are not as developed as in~\gls{AE}. 
%The firm-specific advantages of many emerging economy firms are valuable only in their home country and may also not be sustainable~\cite{chittoor:2008}.
Western \mne in contrast have superior resources and capabilities to adapt their competitive strategies in the face of institutional upheaval~\citep{Chittoor:2008cj,Newman:2000fc,Prahalad:2003th}. 
The ability to adapt is engrained longer in their company DNA and therefor the response comes more natural~\citep{Chittoor:2009jh}.\\
The reaction of especially \gls{EE} \mne to fight the changes is rather futile since firms have to adhere to the agreements that have been agreed upon during the formulation phase~\citep{Lawton:2005wo}.
Only when the agreements have been implemented and the enforcement phase is reached, firms have the option, (through their respective governments) to make amendments to the agreements~\citep{Lawton:2005wo} (see also table~\ref{tab:lawton:2009}).\\
So the responses, to changes in the institutional environment and in this case, change to \wto~\rr~are limited to either product, spatial or internal responses.\\

\begin{subtheorem}{WP}  
 \begin{WP}\label{WP:AE_lower_inst_cha}
Advanced economy firms are expected to face a lower degree of institutional change compared to emerging economy firms with respect to (changes in) WTO \rr 
 \end{WP}
  \begin{WP}\label{WP:greater_diff_heterogeneous_responses}
    The greater the difference in institutional change between regions in response to WTO \rr the more likely heterogeneous responses are from firms in the same industry in their respective regions
  \end{WP}
\end{subtheorem}  

\subsection{Firm responses to WTO \rr}\label{sec:firm_responses_WTO}
%%%%%%%%%%%%%% Changes en de WTO

Looking at firm reactions, one can delve even further into the specific options that are available to firms to adapt or change.
Firms have the opportunity to respond to changes in a number of ways:
In their paper~\cite{Lawton:2009vw} examined, specifically, the types of changes firms make in the wake of changes initiated by the \wto. 
They concluded that firms can (a) adjust their product; (b) initiate a spatial adjustment; or (c) make an internal adjustment~\citep{Lawton:2005wo,Lawton:2009vw}.
~\cite{Chittoor:2008cj} adds to this to divest and exit that specific business.
Where as~\cite{Chittoor:2008cj} provide the fifth option to ``a defensive strategy aimed at defending, protecting and consolidating the position'' or as~\cite{Lawton:2005wo} calls it rule adjustment (harmonisation of trade rules to eliminate regulatory arbitrage).
The responses that firms can have are not mutually exclusive as one can follow the other~\citep{Chittoor:2008cj,Lawton:2009vw}.

\subsubsection{Internal adjustment}
This internal adjustment can also be described as a process where firms aligne their business activities with the multilateral trade rules~\citep{Lawton:2009vw}.
Internal adjustment can be found in, for example, increasing productivity. 
The possibilities on how to achieve this increased productivity stretches beyond the scope of this thesis.\\
The process can be seen as a specialised form of (internal) adoption described according to~\citep{Cantwell:2009hg}.

\subsubsection{Spatial adjustment}
Spatial adjustment can be achieved by for example moving plants, through foreign direct investment, outsourcing, or alliances. 
The Indian government’s agreement to~\gls{trips} forced a restructuring of the Indian industry~\citep{Lawton:2009vw}.\\
The process can also be seen as a specialised form of (internal) adaption according to~\citep{Cantwell:2009hg,Hoekman:2004wa}.
This meant that the old was out and new ways to remain competitive.
The new patent regime (the Indian legislation incorporating \gls{trips}) has also made it imperative that for its sustained future growth, the Indian \pharma industry has to undertake its own innovative research into~\gls{nce} and~\gls{ndds}~\citep{Lawton:2009vw}.\\

\subsubsection{Product adjustment}
Product adjustment provides firms the opportunity to switch out of some product lines and into others~\citep{Ferreira:2010tp} sums this up nicely in stating that product adjustment (or adaption) is found in the form of expansion into new product-markets, including perhaps different customers, the exploration of new market opportunities and possibly development of new resources to tap into the new market.\\
One can observe a differentiation in the kind of response firms have to institutional change. 
This response could be dependant on the type of industry.
It could be imagined that firms have large investments in a plant and thus are less likely to opt for a spatial change in the form of moving a plant to a different country with more favourable (say patent) laws.\\ 
Combining a coevolution and an adjustment strategy has been observed and defined by~\citep{Cantwell:2009hg}.
By changing the product, the entire environment the firm is operating in, may well change.
This might be the case for suppliers and other firms that deliver to that company.
companies and local authorities can cooperate in creating incentives for sustainable R\&D activities~\citep{Bloomberg:2006,Deloitte:2013}.
This knowledge can lead to the following \wpro:

\begin{WP}\label{adjustement_heterogeneous}
  The preferred mode of adjustment to WTO~\rr for MNE in different industries is likely to be heterogeneous
\end{WP}
