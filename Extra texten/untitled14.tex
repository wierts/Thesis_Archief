As shown in Figure 2, strategic choices are not only driven by industry conditions and firm capabilities, but are also a reflection of the formal and informal constraints of a particular institutional framework that managers confront (Jar- zabkowski, 2008)



IBV Critiques

The critiques fall into eight categories: 
(a) the RBV has no managerial implications, 
(b) the RBV implies infinite regress, 
(c) the RBV’s applicability is too limited, 
(d) SCA is not achievable, 
(e) the RBV is not a theory of the firm, 
(f) VRIN/O is neither necessary nor sufficient for SCA, 
(g) the value of a resource is too indeterminate to provide for useful theory, and 
(h) the definition of resource is unworkable.


 We argue below that the first five cri- tiques do not really threaten the RBV’s status. They are incorrect or irrelevant or apply only when the RBV is taken to its logical or impractical extreme; better demarcating the RBV and its variables can contain them. However, the last three critiques offer more serious chal- lenges that need to be dealt with if the RBV is to more fully realize its potential to explain SCA, especially beyond predictable, stable environments.
 


Institutional based view in DE

Numerous studies have used the institution-based view to explain and predict the firm‘s internationalization behavior in the context of developed economies, i.e., DE (e.g., Coeurderoy and Murray, 2008; Descotes et al., 2011; Hessels and Terjesen, 2010),

%%%%%%%%%%%%%%%%%%%%%%%%%%%%%%%%%%%%%%%%%%%%%%%%%%%%%%%%
%       WTO                          %%%%%%%%%%%
\subsection{Trade Agreements}
Before the time of the \wto agreements have been reached during the \gls{gats} era. The results of these was a number of agreements regarding the trading of Goods during 1947--1994 trade talk rounds. Not only were agreements reached on trading of goods but also on lower customs duty rates and other trade barriers.

Since the inception of the \wto the new rules have been committed in the \gls{gatt}. 







McMillan
Dit is de html-versie van het bestand http://faculty-gsb.stanford.edu/mcmillan/personal_page/documents/Market%20Institutions.pdf.
Google maakt automatisch een html-versie van documenten bij het indexeren van het web.
Page 1
