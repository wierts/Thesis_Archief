\chapter{Advanced Economies}\label{ch:AE}

 


Below the countries that make up the Worlds Advanced Economies are listed according to their well known subgroups as defined by the \cite{InternationalMonetaryFund:2013vn}
The table is split into tree parts. 
Some countries are listed twice.
This is because they are both members of the Euro Area and the G7.


\begin{table}\caption[Countries that constitute the Advanced Economies]{Countries that constitute the Advanced Economies source:~\cite{InternationalMonetaryFund:2013vn} }
  \centering
  \begin{tabular}{ccc} 
\multicolumn{3}{c}{\textbf{Euro Area}}          \\
\toprule
Austria     & Germany     & Greece \\
Ireland     & Italy       & Luxembourg \\
Malta       & Belgium     & Cyprus \\
Estonia     & Finland     & France\\
Netherlands & Portugal    & Slovak Republic\\
Slovenia    & Spain       & 
\medskip\\
\multicolumn{3}{c}{\textbf{Major Advanced Economies or Group of Seven (G7)}} \\
\midrule
Canada       & France & Germany\\
Italy        & Japan  & United Kingdom\\
United States&        &\medskip\\
\multicolumn{3}{c}{\textbf{Other Advanced Economies} }\\
\midrule
Australia     & Czech Republic & Denmark\\
Hong Kong SAR\tablefootnote{On July 1, 1997, Hong Kong was returned to the People’s Republic of China and became a Special Administrative Region of China.}
 & Iceland        & Israel\\
Korea         & New Zealand    & Norway\\
San Marino    & Singapore      & Sweden\\
Switzerland   & Taiwan Province of China & \\
\bottomrule
 \end{tabular}
\end{table}



\chapter{Emerging Economies}

The IMF categorises all other countries as Emerging market and developing economies.
According to the IMF these are 153 countries~\cite{InternationalMonetaryFund:2013vn}
in the \gls{WEO} of 2012 a list of what the IMF considers EM economies has been provided.
The countries listed are:\\
Argentina, Brazil, Bulgaria, Chile, China, Colombia, Estonia, Hungary ,
India, Indonesia, Latvia, Lithuania, Malaysia, Mexico, Pakistan, Peru,
Philippines, Poland, Romania, Russia, South Africa, Thailand, Turkey ,
Ukraine, and Venezuela~\cite{IMF:2012}


The Worldbank provides a different scope. 
They measure the \gls{GNI} per capita\footnote{for the calculation method of the \gls{GNI} refer to \url{http://data.worldbank.org/about/data-overview/methodologies}} and determine accordingly the categorisation shown in table \ref{tab:GNI}.



\begin{table}\caption{GNI income distribution source~\cite{WorldBank:2013Classification}}\label{tab:GNI}
    \centering
  \begin{tabular}{ccc}
   \textbf{Income range } & \textbf{\gls{GNI} per capita (in \$ US)}& \textbf{Equivalent Type of Economy}\\
   \toprule
    Low-income            & 1,035 or less        & Developing Economy\\
    Lower-middle-income\tablefootnote{India is considered as Lower-Middle-Income in this characterisation.\cite{WorldBank:2013Country}}   & 1,036 to 4,085       & Developing Economy\\
    Upper-middle-income   & 4,086 to 12,615      & Emerging Markets\tablefootnote{The distinction of Emerging Market Economies is not specifically given in the world bank documentation. However based on the list of countries that is included, such as Eastern European countries, China, Brazil and South Africa, the author feels the terminology is justified.}\\
    High-income           & 12,616 or more       & Advanced Econuomy\\
    \bottomrule
  \end{tabular}
\end{table}

The list of countries that are included in the High-Income range is larger than the list that the IMF is keeping. 
In the classification that the worldBank uses, countries such as Brunei Darussalam, the Russian Federation, Oman, Saudi Arabia and Chile are seen as part of the High-Income group~\cite{WorldBank:2013Country}.\\
A number of Banks have made their classification as well. 
In the landmark piece on Emerging Markets~\cite{ONeill:2001wa} the term BRICs was coined as an acronym introducing Brazil, Russia, India and China as the leasing Emerging Markets. 
Others have also identified India as an emerging market.

\newcommand{\tick}{\large\ding{51}}
\newpage

\begin{table}
  \centering
  \caption{Emerging Market Economies according to different sources}\label{tab:EE}
    \begin{tabular}{lcccccc} \textbf{Institute}     & IMF  & FTSE\tablefootnote{\cite{FTSE:2012to}}  &  MSCI~\tablefootnote{\cite{MSCI:2013}}  &  S\&P Dow Jones\tablefootnote{\cite{StandardandPoors:2013vd}}   &  EMGP\tablefootnote{the Emerging Market Global Player project is maintained by the Columbia University \url{http://www.vcc.columbia.edu/content/emerging-market-global-players-project}} \\
 \textbf{Country}       &      &       &        &              &    \\    
 \toprule
    Argentina           &\tick & ~     & ~      & ~            & \tick                       \\
    Brazil              & \tick& \tick & \tick  & \tick        & \tick                        \\
    Bulgaria            & \tick& ~     & ~      & ~            & ~                        \\
    Chile               & \tick& \tick & \tick  & \tick        & \tick                        \\
    China		       & \tick& \tick & \tick  & \tick        & \tick                        \\
    Columbia            & ~    & \tick & \tick  & \tick        & ~                        \\
    Czech Republic      & ~    & \tick & \tick  & \tick        & ~                        \\
    Egypt		       & ~    & \tick & \tick  & \tick        & ~                        \\
    Estonia			   & \tick& ~     & ~      & ~            & ~                        \\
    Hungary             & \tick& \tick & \tick  & \tick        & \tick                        \\
    India	           &\tick & \tick & \tick  & \tick        & \tick                        \\
    Indonesia		   &\tick & \tick & \tick  & \tick        & ~                        \\
    Latvia		       &\tick & ~     & ~      & ~            & ~                        \\
    Lithuania	       &\tick& ~     & ~      & ~            & ~                        \\
    Malaysia            & \tick& \tick & \tick  & \tick        & ~                        \\
    Mexico		       & \tick& \tick & \tick  & \tick        & \tick                        \\
    Morocco			   & ~    & \tick & \tick  & \tick        & ~                        \\
    Pakistan	           & \tick & \tick & ~      & ~           & ~                        \\
    Peru    	           & \tick & \tick & \tick  & \tick       & ~                        \\
    Philippines		   & \tick & \tick & \tick  & \tick       & ~                        \\
    Poland		       & \tick & \tick & \tick  & \tick       & \tick                        \\
    Romania			   & \tick & ~     & ~      & ~           & ~                        \\
    Russia			   & \tick & \tick & \tick  & \tick       & \tick                        \\
    South Africa	       & \tick & \tick & \tick  & \tick        & ~                      \\
    Taiwan		       & ~     & \tick  & \tick & \tick        & \tick                        \\
    Thailand		       & \tick  & \tick & \tick & \tick        &                         \\
    Turkey		       & \tick  & \tick & \tick & \tick        & \tick                        \\
    UAE				   & ~     & \tick  & ~     & ~            & ~                        \\
    Ukraine			   & \tick & ~      & ~     & ~            & ~                        \\
    Venezuela		   & \tick & ~      & ~     & ~            &  ~        \\ 
 \bottomrule 
   \end{tabular}
\end{table}


