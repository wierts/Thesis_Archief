Both theories of Porter and Barney made their respective impressions in Strategic literature.  However~\gls{RBV}, which in introspective in nature and \gls{InBV} is extrospective cannot account for the entire spectrum of conscious decision making.
Considering the theories of Barney and Porter the Dutch `Plofkip'\footnote{Loosely translated in English as BangChick} would be a highly successful product.\\
In the 2010s in the Netherlands a new breed in chicks (a broiler) was bread. From a scientific standpoint this was a very successful product with regard to sustainability and food conversion. 
This `plofkip' as it was rebranded by the media and animal welfare organisations has certain traits that are very positive\footnote{Source taken from websites (\url{http://fransvdst.wordpress.com/2012/10/28/het-duurzame-vleeskuiken})}\footnote{Source Dutch newspaper `Trouw'
(\url{http://www.trouw.nl/tr/nl/5948/Dierenwelzijn/article/detail/3267278/2012/06/07/De-voordelen-van-de-plofkip.dhtml}) 
and (\url{http://www.trouw.nl/tr/nl/4332/Groen/article/detail/3324936/2012/10/01/Vanuit-het-milieu-gezien-is-er-niets-mis-met-plofkip-en-megastal.dhtml})}
These broilers had noticeable advantages:

\begin{itemize}
 \setlength{\itemsep}{0.75pt}
\item The broiler grows faster than the free-range chicken's carcass (40 days vs. 56 days) and needs to be fed far less and therefore less manure needs to be removed
 \item    The broiler grows better with less food than the free-range chicken (has better feed conversion). Thus the free-range chicken needs more food to gain one kilogram of meat.
\item     The broiler has less stable space at his disposal including what should be heated
\item     The manure of a broiler remains in the stable and is thus more environmentally friendly processes
  \item   The broiler is more tender and cooks faster
\end{itemize}
Despite all these advantages, the disadvantages (cleverly supported by pictures) that the chickens had such difficulties as supporting their own bodyweight and their harts having issues coping with the high growth rate~\footnote{Source taken form \url{http://www.wakkerdier.nl/actueel/plofkip-campagne and translated freely by the author}}, the broiler had to be abandoned and is no longer `in play' as a food source.\\
The aforementioned example can not be explained by the theories by~\cite{Porter:1980,Barney:1991}, they cannot explain for this phenomenon. 
The lack of context in with this product has to `operate'  and the power of institutions are not catered for in the theories of Barney and Porter.