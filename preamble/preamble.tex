\usepackage{setspace}% benodigd voor spacing commando
\usepackage[nottoc]{tocbibind}
\usepackage{titlesec}
%\usepackage{eucal}
\usepackage[usenames,dvipsnames]{xcolor} %
\usepackage{datetime} % gebruikt voor \pdfdate om juiste datum weergave te krijgen
\usepackage{fancyhdr}
\usepackage{graphicx}%[pdftex]
\usepackage[utf8]{inputenc} %f ̧r Windows %direct gebruik van accenten
\input{ix-utf8enc.dfu}% dit is om u ̃ te laden
%\usepackage[T1]{fontenc}
\usepackage{eurosym,tfrupee,supertabular} %tfrupee om het rupee symbool te maken 
\usepackage[a4paper]{geometry}
\usepackage{rotating}
%\usepackage{parskip}
\usepackage[small,bf,up]{caption}%[2004/07/16]
\usepackage{tablefootnote}
\usepackage{amsthm}% package om te assisteren bij Working Proposistions
\usepackage{url} % om urls goed weer te geven in voetnoten
\usepackage[toc,page,title]{appendix}
\usepackage[multiple,flushmargin,stable]{footmisc}
\usepackage{datatool} %databases laden in latex
\usepackage{booktabs}% om mooie tabellen te maken
%\usepackage{hanging}% provides the \hangpara command
%\usepackage{enumitem}
\usepackage{enumerate}
\usepackage{float} %gebruikt om floatplacement te activeren
\usepackage{tabularx,ragged2e} %in samenwerking met tabularx
\newcolumntype{Y}{>{\RaggedRight\arraybackslash}X} % modified "X" column
\usepackage{multirow}%
\RequirePackage[l2tabu, orthodox]{nag}
\usepackage{fixltx2e}% moet enkele latex fouten corrigeren
\usepackage{tikz} % tekenen van plaatjes
\usepackage{xspace} %correcte 'spaties' achter en voor comments en punten/comma'
\usepackage[protrusion=true,expansion]{microtype} %spacing tussen letters enzo 
\usetikzlibrary{arrows,shapes,decorations.pathmorphing,backgrounds,fit,positioning,shapes.symbols,chains}
\usetikzlibrary{external}
\tikzexternalize[prefix=TikzFigures/]
\tikzexternalize
%\usepackage{illcmolthesis} %dit maakt de titel pagina

\doublespacing%

    \definecolor{gray75}{gray}{0.75}
    \DeclareGraphicsExtensions{.png, .jpg, .pdf}%
    \pdfcompresslevel=9
    \graphicspath{{./Figures/}}

%\DeclareUnicodeCharacter{0169}{\~u} % make the character known
%\DeclareUnicodeCharacter{226}{\^a}

%--------------Layout van A4 document-----------------
\geometry{a4paper,
  %includefoot,
 % margin=2.5cm,
  %hdivide={ ,19cm, }  
}

%------------------Footnotes--------------------------
%\renewcommand{\footnotemargin}{1em}% onderdeel van footnotemisc
% changes the above and sets the footnote mark just right of the left margin border.

%\newcommand{\fn}[1]{\footnote{\hangpara{3em}{1} #1}}
% makes a new footnote command \fn{} with a hanging indent of 3em (hanging indent starts after the first line)

%------------Biblatex Optie 1---------------------------

%%\usepackage[backend=biber,style=authoryear-luh-ipw,isbn=false,url=false]%{biblatex} ,style=historian ,citestyle=apa
\usepackage[
backend=biber,
citestyle=authoryear-icomp,
bibstyle=authortitle,
sorting=nyt,
bibencoding=utf8,
dashed=false,
maxcitenames=1,
isbn=false,
url=false,
babel=hyphen,
hyperref=true,
doi=false]
{biblatex}

\let\cite\textcite


\renewcommand*{\nameyeardelim}{\addcomma\space}
\renewcommand*{\postnotedelim}{\addcolon\space}
\DeclareFieldFormat{postnote}{#1}
\DeclareFieldFormat{multipostnote}{#1}

\usepackage{xpatch}
\xpretobibmacro{date+extrayear}{\addperiod\space}{}{}
\xapptobibmacro{date+extrayear}{\nopunct}{}{}


% Citation Hyperlinks (not just years), thanks to Audrey.
\makeatletter
\renewbibmacro*{cite}{% Based on cite bib macro from authoryear-comp.cbx
  \iffieldundef{shorthand}
    {\ifthenelse{\ifnameundef{labelname}\OR\iffieldundef{labelyear}}
       {\printtext[bibhyperref]{% Include labelname in hyperlink
          \DeclareFieldAlias{bibhyperref}{default}% Prevent nested hyperlinks
          \usebibmacro{cite:label}%
          \setunit{\addspace}%
          \usebibmacro{cite:labelyear+extrayear}}%
          \usebibmacro{cite:reinit}}
       {\iffieldequals{namehash}{\cbx@lasthash}
          {\ifthenelse{\iffieldequals{labelyear}{\cbx@lastyear}\AND
                       \(\value{multicitecount}=0\OR\iffieldundef{postnote}\)}
             {\setunit{\addcomma}%
              \usebibmacro{cite:extrayear}}
             {\setunit{\compcitedelim}%
              \usebibmacro{cite:labelyear+extrayear}%
              \savefield{labelyear}{\cbx@lastyear}}}
          {\printtext[bibhyperref]{% Include labelname in hyperlink
             \DeclareFieldAlias{bibhyperref}{default}% Prevent nested hyperlinks
             \printnames{labelname}%
             \setunit{\nameyeardelim}%
             \usebibmacro{cite:labelyear+extrayear}}%
             \savefield{namehash}{\cbx@lasthash}%
             \savefield{labelyear}{\cbx@lastyear}}}}
    {\usebibmacro{cite:shorthand}%
     \usebibmacro{cite:reinit}}%
  \setunit{\multicitedelim}}
%--------Biblatex-------------------------------------

%\DeclareNameAlias{sortname}{first-last}

%\DeclareCiteCommand{\cite}[\mkbibbrackets]
 % {\usebibmacro{prenote}}
 % {\usebibmacro{citeindex}%
  % \usebibmacro{cite}}
%  {\multicitedelim}
%  {\usebibmacro{postnote}}

%\DeclareCiteCommand*{\cite}[\mkbibbrackets]
%  {\usebibmacro{prenote}}
 % {\usebibmacro{citeindex}%
%   \usebibmacro{citeyear}}
%  {\multicitedelim}
 % {\usebibmacro{postnote}}


%\newcounter{mymaxcitenames}
%\AtBeginDocument{%
 % \setcounter{mymaxcitenames}{\value{maxnames}}%
%}

%\renewbibmacro*{begentry}{%
 % \printtext[brackets]{%
  %  \begingroup%
 %   \defcounter{maxnames}{\value{mymaxcitenames}}%
  %  \printnames{labelname}%
 %   \setunit{\nameyeardelim}%
 %   \usebibmacro{cite:labelyear+extrayear}%
 %   \endgroup%
  %  }%
%  \quad% or \addspace
%}

%--------END Biblatex-------------------------------------



%------------Biblatex Optie 2---------------------------

\usepackage[
        natbib=true,
        citestyle=authoryear-comp,
        bibstyle=authoryear,
        hyperref=false,
        backend=biber,
        %bibencoding=auto,
        maxbibnames=99,
        uniquename=false,
        maxcitenames=1,
        dashed=false,
        url=false,
        doi=false,
        isbn=false,
        eprint=false
                    ]{biblatex}
 
\renewcommand{\cite}{\textcite} 

% remove "in:" from articles. Thanks to Herbert.
\renewbibmacro{in:}{%
  \ifentrytype{article}{}{%
  \printtext{\bibstring{in}\intitlepunct}}}

% increase vertical space between bibliography items.
\setlength\bibitemsep{0.5ex}
\setlength\bibnamesep{1.2ex}

% makes volume of journal bold and adds colon
\DeclareFieldFormat[article]{volume}{\textbf{#1}\addcolon\space}


%\renewcommand*{\textcitedelim}{\addcomma\space}% if you want another delimiter

%\newcommand*{\mywrapper}[1]{%
%    \ifthenelse{\value{textcitetotal}>1}
%    {\mkbibbrackets{#1}}
%    {#1}}
%
%\makeatletter
%\DeclareMultiCiteCommand{\cbx@textcites}[\mywrapper]{\cbx@textcite}{} 
%
%\DeclareCiteCommand{\cbx@textcite}[\mywrapper]
%  {\usebibmacro{cite:init}}
%  {\usebibmacro{citeindex}%
%   \usebibmacro{textcite}}
%  {}
%  {\usebibmacro{textcite:postnote}}
%
%\makeatother


% Dit is de juiste definitie voor het gebruik van de bibs
%----------------------------CSQuotes na Biblatex
\usepackage[babel]{csquotes}%
\usepackage[english]{babel}%
%------------------ FONT ----------------------------------------

 \usepackage{pifont}%gebruik van checkmarkt
 \usepackage{charter}%
% ALTERNATIVE
%\usepackage{lmodern}
%\usepackage{fouriernc}
%\usepackage{MinionPro}
%\usepackage[urw-garamond]{mathdesign}
%\usepackage[math]{iwona}
%\usepackage[default]{lato}
%\usepackage{venturis2}
%\usepackage[sf]{quattrocento}
%\usepackage{libertine}



%--------Hyperref zo laat mogelijk------------------------

\usepackage[   pdftex, 
               %  plainpages = false, 
               %  pdfpagelabels, 
                % pdfpagelayout = OneColumn, 
               %   display single page, advancing flips the page - Sasa Tomic
                % bookmarks,
               %  bookmarksopen = true,
                % bookmarksnumbered = true,
               %  breaklinks = true,
                % linktocpage,
                 colorlinks = false,
                % linkcolor = black,
                % urlcolor  = black,
                % citecolor = black,
                % anchorcolor = black,
                 hyperindex = true,
               %  pageanchor = false,
                % hyperfigures
                 ]{hyperref}% als laatste laden om conflicten te voorkomen 
%----------einde titel opmaak-------------------
\floatplacement{table}{htbp} 
\floatplacement{figure}{htbp}
\setlength{\columnseprule}{\headrulewidth}
\setcounter{tocdepth}{2}

%------------------ END FONT ------------------

\linespread{1.5} %\setlength{\parskip}{\baselineskip}
\setlength{\headheight}{43,3pt}
\setlength{\voffset}{0pt}

\fancypagestyle{plain}{\fancyhf{}\fancyfoot[R]{\thepage}
\renewcommand{\headrulewidth}{0.4pt}
\renewcommand{\footrulewidth}{0.4pt}
}

\pagestyle{fancy}
\fancyhf{}     % clear header & footer
\fancyhead[C]{\small\sffamily\nouppercase{\leftmark}}
%\fancyhead[L]{\includegraphics[width=.19\textwidth]{uva_logo_eng.eps}} %links logo
\fancyhead[R]{\includegraphics[width=.19\textwidth]{uva_ENG.eps}}  %rechts logo
%\fancyfoot[L]{\thepage}
\fancyfoot[R]{\thepage}
\renewcommand{\headrulewidth}{0.4pt}
\renewcommand{\footrulewidth}{0.4pt}

\renewcommand{\captionfont}{\small\textit}

\renewcommand{\headrule}{{\color{gray}%
\hrule width\headwidth%
 height\headrulewidth \vskip-\headrulewidth}
 }

\renewcommand{\footrule}{{\color{gray}% 
  \vskip-\footruleskip\vskip-\footrulewidth%
\hrule width\headwidth height\footrulewidth\vskip\footruleskip}
}
\renewcommand{\chaptermark}[1]{\markboth{\textbf{\thechapter\ #1}}{}}
\renewcommand{\sectionmark}[1]{\markright{\thesection\ #1}{}}
\fancyheadoffset{0cm} % zorgt er voor dat de head en foot lijnen even breed zijn als de textbreedte


%%%%% by default first HYPERREF than Glossaries packgages
\usepackage[acronym,toc,xindy,nomain]{glossaries}% 
\glsdisablehyper%
\makeglossaries%
%____________THEOREM___________

\makeatletter

\newenvironment{subtheorem}[1]{%
  \def\subtheoremcounter{#1}%
  \refstepcounter{#1}%
  \protected@edef\theparentnumber{\csname the#1\endcsname}%
  \setcounter{parentnumber}{\value{#1}}%
  \setcounter{#1}{0}%
  \expandafter\def\csname the#1\endcsname{\theparentnumber.\Alph{#1}}%
  \ignorespaces
}{%
  \setcounter{\subtheoremcounter}{\value{parentnumber}}%
  \ignorespacesafterend
}
\makeatother
\newcounter{parentnumber}


%%%%%%%%%% stukje code voor oplopend versie nummer%%%%%%%%%%%%%

\newcounter{run}
\usepackage{catchfile}
\IfFileExists{\jobname.runs}{%
  \begingroup
    \CatchFileEdef\tmp{\jobname.runs}{\endlinechar=-1\relax}%
    \setcounter{run}{\tmp}%
  \endgroup
}{}
\stepcounter{run}

\usepackage{atveryend}
\usepackage{newfile}
\AtVeryEndDocument{%
  \newoutputstream{runs}%
  \openoutputfile{\jobname.runs}{runs}%
  \addtostream{runs}{\number\value{run}}%
  \closeoutputstream{runs}%
}


%%%%%%%%%%%% aanvulling voor Latex Diff  %%%%%%%%%%%%%%
%\providecommand{\DIFadd}[1]{{\protect\color{blue}#1}} %%%% Dit staat in thesisdw document
%\providecommand{\DIFdel}[1]{{\protect\color{red}#1}  %%%% Dit staat in thesisdw document
%\providecommand{\DIFadd}[1]{{\protect\color{blue}#1}} %DIF PREAMBLE  %%%% Dit staat in thesisdw document
%\providecommand{\DIFdel}[1]{{\protect\color{red}\protect\scriptsize{#1}}}  %%%% Dit staat in thesisdw document