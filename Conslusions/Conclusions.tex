\chapter{Conclusions}
In this chapter, the conclusions regarding the findings will be presented.
Also any limitations of the research shall be discussed as well as the implications and suggestions for additional and future research.

\section{Conclusions}
This research has been investigating the effects of WTO (institutional) \rr changes on \glsfull{AE} and \glsfull{EE} firms.
There is not much novelty in pitching firms from AE and EE regions against each other.
China and the US are prime the usual suspects in this case.
Now Europe and India have been selected to supply the necessary companies to fill our data set.
This research has investigated firms in different industries, the \manu industry through \pharma firms and the services industry through \its firms.\\
When it comes to response to change, firms can employ a number of tactics.
Responses can be found on strategy level (avoidance, adaption and coevolution are available~\citep{Cantwell:2009hg}) and on operational level (Internal, spatial and product adjustment~\citep{Lawton:2009vw,Hoekman:2004wa}).
More than often the responses are a copies~\citep{Scott:2008,DiMaggio:1983wt} of successful changes by firms that form a example.
Sometimes the change or response is a human factor.
Internal resources (as defines by~\cite{Barney:1991ur}) thinking in a certain direction.
This is than adopted by that company.\\
To answer the research question, a \mcs of four European and four Indian (eight in total) has been conducted.
See in Figure~\ref{fig:8firms} how the firms are divided of the the economic regions and industries.

\begin{figure}[!htb]
    \centering
    \tikzsetnextfilename{alle_8_firms}
    \begin{tikzpicture} [
        title/.style={font=\fontsize{18}{18}},
        capt/.style={font=\fontsize{30}{30}},
        firm/.style={rectangle, draw, rounded corners, minimum width=6em, minimum height=6.2em,fill opacity=0.5},
        industry/.style={rectangle, draw, fill=black!23, rounded corners, minimum width=28.95em,fill opacity=0.3},
        region/.style={rectangle, draw, fill=black!23, rounded corners, minimum width=8em,fill opacity=0.3, minimum height=17.5em},
        ]
%         Place nodes
        \node (dummy) {Industry};
        \node [title] (in)  [right=2.55cm of dummy] {India};
        \node [title] (eur) [right=6.03cm of dummy] {Europa}; 
        \node [right=2.01cm of eur] (dummy2) {Industry};
        
        \node [title] (infy) [below=10mm of in] {Infosys};
        \node [title] (cap)  [below=10mm of eur] {CapGemini};
                
        \node [title] (Ts)   [below=20mm of eur] {T-systems};        
        \node [title] (wp)   [below=20mm of in]  {WiPro};
             
        \node [title] (ran) [below=3.75cm of in]  {Ranbaxy};
        \node [title] (cip) [below=4.75cm of in]  {Cipla};
        
        \node [title] (nov) [below=3.75cm of eur]  {Novartis};
        \node [title] (gsk) [below=4.75cm of eur]  {GSK};
        
        
        \node [firm] (InIt) [draw=black!50, fit= (infy) (wp)] {};        
        \node [firm] (EuIt) [draw=black!50, fit= (cap) (Ts)] {};
  
        \node [capt] (its) [anchor=west,below=1.5cm of dummy] {IT Services};
        \node [capt] (ph)  [anchor=east,below=4.25cm of dummy2] {Pharmaceutical};
        
        \node [industry][draw=black!50, fit=(its)(InIt)(EuIt)]{}; 
               
        \node [firm] (EuPh) [draw=black!50, fit= (nov) (gsk)]  {};
        \node [firm] (InPh) [draw=black!50, fit= (ran) (cip)]  {};
        
        \node [region] [draw=black!50, fit=(eur)(EuPh)]{}; 
        \node [region] [draw=black!50, fit=(in)(InPh)]{}; 
        \node [industry] [draw=black!50, fit=(ph)(InPh)(EuPh)]{}; 
        
    \end{tikzpicture}
    \caption{Firms per Industry and Region}\label{fig:8firms}
\end{figure}


The results of the analysis show, EE are influenced more by changes in WTO \rr than their AE cousins.
Especially the EE \pharma firms faced significant institutional change.
To overcome this change they has to change~\citep{Lawton:2009vw,Cantwell:2009hg} or challenge~\citep{Bartlett:1989vl} the existing \rr.\\
The history of the firm~\citep{Chittoor:2008cj} and the internal resources~\citep{Barney:1991ur} (in the form of their leadership) were influenced by the choices that the firms made, facing the new (institutional) reality.
The multinational character of the investigated firms had a harmonising effect on their responses.

\section{Limitations}
This study has been conducted by analysing newspapers articles from various large mostly english language newspapers. %This has worked for the Indian and English companies.
However the amount of sources for the Swiss, German and French companies has been limited.
Some German language newspapers have been consulted. 
This increased the number of sources to some extend.
However it did not provide the same number of sources as was available for the Indian and British companies.
This restricted the information availability.
On the other hand the number of sources that were available on especially the Indian companies was very large.
This rendered a keyword selection necessary to limit the amount of `hits'.
Only two newspapers have been used with one keyword. 
This limits the information that has been found and also limits the diversity of the information sources.
Using keywords, made the sources biased to some extend towards those keywords.
The nature and origin of search engines influences the data that is used.
Better quality data is generated when using multiple search engine sources~\citep{Torralba:2011uw}.
He found that ``Obtaining data from multiple sources (e.g. multiple search engines from multiple countries) can somewhat decrease selection bias''~\citep[p.1527]{Torralba:2011uw}.

\section{Implications}
This study can contribute in two ways.
First it shows that what is equal on paper might not always equal in its effects.
The WTO \rr are equal for \glsfull{AE} and \glsfull{EE} alike.
The responses from the companies in those economies are not.
Taking into account the location of a company is of influence to the response direction of that company.
This insight can help in better formulating new \rr.\\
Secondly it can also be a stepping stone for the formulation of future WTO \rr.
This study give the insight that, in different economic regions WTO \rr are effecting firms differently.
When considering future \rr, the implications mentioned in this study can be helpful in formulation more balanced \rr towards advanced and emerging economies.

\section{Future Research}

Further research can be done by extending the newspaper search to include sources from the home countries of the companies involved in the study.
Most hits have been found when the newspaper and the company have the same home country.
Finding information in Swiss, German and French newspapers in their respective companies is suggested.
Furthermore seeking more diverse sources on the indian companies omitting keywords when finding sources can broaden the scope of information.\\
This research has only chosen a subset of \manu and services companies.
To increase the quality, a larger set of companies within the subsets of \manu and services companies is suggested.
The extending to more than two companies increases the change of finding outliers in the information set.
Cipla in their, somewhat activist stance, could be one of those outliers.\\
This study did not take the maturity of the companies or the industry into account.
The analysis gives some indication that maturity of the industry and companies in that industry can have an implication on the type and direction of the response towards the industry \cc.
